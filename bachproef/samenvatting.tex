%%=============================================================================
%% Samenvatting
%%=============================================================================

% TODO: De "abstract" of samenvatting is een kernachtige (~ 1 blz. voor een
% thesis) synthese van het document.
%
% Deze aspecten moeten zeker aan bod komen:
% - Context: waarom is dit werk belangrijk?
% - Nood: waarom moest dit onderzocht worden?
% - Taak: wat heb je precies gedaan?
% - Object: wat staat in dit document geschreven?
% - Resultaat: wat was het resultaat?
% - Conclusie: wat is/zijn de belangrijkste conclusie(s)?
% - Perspectief: blijven er nog vragen open die in de toekomst nog kunnen
%    onderzocht worden? Wat is een mogelijk vervolg voor jouw onderzoek?
%
% LET OP! Een samenvatting is GEEN voorwoord!

%%---------- Nederlandse samenvatting -----------------------------------------
%
% TODO: Als je je bachelorproef in het Engels schrijft, moet je eerst een
% Nederlandse samenvatting invoegen. Haal daarvoor onderstaande code uit
% commentaar.
% Wie zijn bachelorproef in het Nederlands schrijft, kan dit negeren, de inhoud
% wordt niet in het document ingevoegd.

\IfLanguageName{english}{%
\selectlanguage{dutch}
\chapter*{Samenvatting}
\lipsum[1-4]
\selectlanguage{english}
}{}

%%---------- Samenvatting -----------------------------------------------------
% De samenvatting in de hoofdtaal van het document

\chapter*{\IfLanguageName{dutch}{Samenvatting}{Abstract}}

Bij water- en wintersporters is er een nood aan een budgettaire, gebruiksvriendelijke, waterdichte en betrouwbare GPS-tracker. Deze toestellen kunnen voor allerlei situaties gebruikt worden zoals uitrusting terugvinden en als lawinepieper. Maar deze toestellen zijn duur, waardoor winter- en watersporters vaker geneigd zijn deze niet aan te schaffen. Hieruit vloeit het onderzoek naar een economisch rendabel GPS-toestel dat niet bezwijkt onder ongunstige omstandigheden. Tijdens het onderzoek is er onderzocht welke locatiebepalingstechnologieën er gebruikt kunnen worden en welke eisen voor de hardware voldaan moeten worden. Hieruit bleek dat een MKR GSM module van Arduino een ideale kandidaat is in combinatie met een MKR GPS SHIELD. De gebruikte hardware kan gebruik maken van het Standard Positioning System (SPS) en General Packet Radio Service (GPRS) als locatiebepalingstechnologieën. De proof of concept is instaat om makkelijk zijn locatie te delen aan de hand van een webapplicatie. Hierdoor kan het terugvinden van iets of iemand slaagt voor alle onderzochte deelonderzoeksvragen. Helaas blijven twee deelonderzoeksvragen onbeantwoord dankzij de corona pandemie. De getrokken conclusie is dat het mogelijk is om een economisch rendabel GPS-toestel te ontwikkelen dat niet bezwijkt onder ongunstige omstandigheden, aangezien alle deelonderzoeksvragen geslaagd zijn.