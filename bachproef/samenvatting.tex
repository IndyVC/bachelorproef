%%=============================================================================
%% Samenvatting
%%=============================================================================

% TODO: De "abstract" of samenvatting is een kernachtige (~ 1 blz. voor een
% thesis) synthese van het document.
%
% Deze aspecten moeten zeker aan bod komen:
% - Context: waarom is dit werk belangrijk?
% - Nood: waarom moest dit onderzocht worden?
% - Taak: wat heb je precies gedaan?
% - Object: wat staat in dit document geschreven?
% - Resultaat: wat was het resultaat?
% - Conclusie: wat is/zijn de belangrijkste conclusie(s)?
% - Perspectief: blijven er nog vragen open die in de toekomst nog kunnen
%    onderzocht worden? Wat is een mogelijk vervolg voor jouw onderzoek?
%
% LET OP! Een samenvatting is GEEN voorwoord!

%%---------- Nederlandse samenvatting -----------------------------------------
%
% TODO: Als je je bachelorproef in het Engels schrijft, moet je eerst een
% Nederlandse samenvatting invoegen. Haal daarvoor onderstaande code uit
% commentaar.
% Wie zijn bachelorproef in het Nederlands schrijft, kan dit negeren, de inhoud
% wordt niet in het document ingevoegd.

\IfLanguageName{english}{%
\selectlanguage{dutch}
\chapter*{Samenvatting}
\lipsum[1-4]
\selectlanguage{english}
}{}

%%---------- Samenvatting -----------------------------------------------------
% De samenvatting in de hoofdtaal van het document

\chapter*{\IfLanguageName{dutch}{Samenvatting}{Abstract}}

Water- en wintersporters hebben nood aan een gebruiksvriendelijke, waterdichte en betrouwbare GPS-tracker. Deze toestellen kunnen voor allerlei situaties gebruikt worden zoals het terugvinden van uitrusting en als lawinepieper. Vooralsnog zijn de bestaande toestellen vrij duur, waardoor de barrière voor water- en wintersporters hoger is om een dergelijk toestel aan te schaffen. Er dringt zich dus een onderzoek op naar de economische rendabiliteit van een GPS-toestel dat niet bezwijkt onder ongunstige omstandigheden. In dit onderzoek wordt onderzocht welke locatiebepalingstechnologieën er gebruikt kunnen worden en aan welke eisen de hardware moet voldoen. Hieruit blijkt dat de MKR GSM module in combinatie met de MKR GPS SHIELD de beste resultaten oplevert. Geschikte locatiebepalingstechnologieën zijn het Standard Positioning System (SPS) en General Packet Radio Service (GPRS). De proof of concept die ontwikkeld werd, is in staat om diens locatie te delen aan de hand van een zelf ontwikkelde webapplicatie. De proof of concept voldoet aan alle eisen die werden uiteengezet in de probleemstelling. De coronapandemie heeft de initiële reikwijdte van het onderzoek enigszins beperkt. Het was door het verbod op niet-essentiële verplaatsingen niet mogelijk om de invloed van sneeuw en zout water te achterhalen. De andere onderzoeksvragen konden wel beantwoord worden aan de hand van een experimenteel design. Er kan geconcludeerd worden dat het mogelijk is om een economisch rendabel toestel te ontwikkelen dat niet bezwijkt onder ongunstige omstandigheden en voldoet aan alle kwaliteitsvereisten.