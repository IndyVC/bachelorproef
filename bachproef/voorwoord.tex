%%=============================================================================
%% Voorwoord
%%=============================================================================

\chapter*{\IfLanguageName{dutch}{Woord vooraf}{Preface}}
\label{ch:voorwoord}
%% TODO:
%% Het voorwoord is het enige deel van de bachelorproef waar je vanuit je
%% eigen standpunt (``ik-vorm'') mag schrijven. Je kan hier bv. motiveren
%% waarom jij het onderwerp wil bespreken.
%% Vergeet ook niet te bedanken wie je geholpen/gesteund/... heeft
Eerst en vooral wil ik mijn familie bedanken voor alle steun tijdens mijn hele opleiding. De studie was zeker en vast niet makkelijk, maar ze stonden altijd paraat wanneer ik ze nodig had. Hiervoor, een welgemeende dankjewel.
\newline
\newline
Ook wil ik mijn promotor, Jens Buysse, bedanken voor het aanreiken van het interessante onderwerp. Het onderwerp is zeer 'Internet of Things' gerelateerd, wat mooi aansluit bij mijn interesses en afstudeerrichting. Ikzelf ben ook een snowboarder en surfer, waardoor ik zelf ook het belang inzie van een economisch rendabele GPS-tracker. Ik hoop dat mijn werk gebruikt kan worden in de winter- en watersport.
\newline
\newline
Graag bedank ik ook mijn co-promotors Kevin DeRudder en Thomas Pollet voor de constructieve feedback.
\newline
\newline
Tijdens deze bachelorproef kon ik proeven van  verschillende vakgebieden zoals het ontwerpen van een backend in Mongoose, het ontwikkelen van een webapplicatie in Vue.JS en een cross-platform applicatie. Ik heb mij deze vaardigheden tijdens de proef eigen gemaakt. 
\newline
\newline
Verder hoop ik dat deze bacherlorproef een verrijking is voor de lezer en hem eventueel kan inspireren tot het ontwikkelen van loT devices.