\chapter{\IfLanguageName{dutch}{Broncode Arduino}{Source code Arduino}}
\label{ch:broncode_arduino}
\begin{lstlisting}
#include <ArduinoHttpClient.h>
#include <MKRGSM.h>
#include <ArduinoJson.h>
#include <Arduino_MKRGPS.h>

// PIN Number
const char PINNUMBER[]     = ""; //blank if no pin
// APN data: check settings on mobile for sim provider or contact carrier for network settings
const char GPRS_APN[]      = "telenetwap.be"; //This is for Telenet
const char GPRS_LOGIN[]    = ""; 
const char GPRS_PASSWORD[] = "";


GSMClient client;
GPRS gprs;
GSM gsmAccess;
GSMLocation location;

char server[] = "indy-bap-backend.herokuapp.com";
char path[] = "/api/locations";
int port = 80;

StaticJsonDocument<200> jsonBuffer;
HttpClient httpClient = HttpClient(client, server, port);
JsonObject root = jsonBuffer.to<JsonObject>();
String response;
int statusCode = 0;
String dataStr;

void setup() {
	// initialize serial communications and wait for port to open:
	Serial.begin(9600);
	while (!Serial) {
		; // wait for serial port to connect. Needed for native USB port only
	}
	
	Serial.println("Starting GSM connection to server!");
	// connection state
	boolean connected = false;
	
	// After starting the modem with GSM.begin()
	// attach the shield to the GPRS network with the APN, login and password
	while (!connected) {
		if ((gsmAccess.begin(PINNUMBER) == GSM_READY) &&
		(gprs.attachGPRS(GPRS_APN, GPRS_LOGIN, GPRS_PASSWORD) == GPRS_READY)) {
			connected = true;
			location.begin();
		} else {
			Serial.println("Not connected");
			delay(1000);
		}
	}
	Serial.println("connecting...");
	// if you get a connection, report back via serial:
	if (client.connect(server, port)) {
		Serial.println("connected to server");
	} else {
		// if you didn't get a connection to the server:
		Serial.println("connection to server failed");
	}
	if(!GPS.begin()){
		Serial.println("Failed to initialize GPS!");
		while(1);      
	}
}

void loop() {
	unsigned long wait = millis();
	bool gps = false;
	while (millis() - wait < 2000 && !gps) {
		if(GPS.available()){
			Serial.println("GPS method");
			root["longitude"] = GPS.longitude();
			root["latitude"] = GPS.latitude();
			root["altitude"] = GPS.altitude();
			root["speed"] = GPS.speed();
			root["satellites"] = GPS.satellites();
			root["method"] = "GPS";
			root["device"] = "poc";
			gps=true;
		}
	}
	if(!gps){
		Serial.println("GPRS method");
		unsigned long timeout = millis();
		while (millis() - timeout < 2000) {
			if (location.available() && location.accuracy() < 300 && location.accuracy() != 0) {
				root["longitude"] = String(location.longitude(),20);
				root["latitude"] = String(location.latitude(),20);
				root["altitude"] = String(location.altitude(),20);
				root["method"] =  "GPRS";
				root["device"] = "poc";
			}
		}
	}
	//if you get a connection, report back via serial:
	if (client.connect(server, port)) {
		postToServer(root);
	} else {
		// if you didn't get a connection to the server:
		Serial.println("connection failed");
	}
	delay(3000); // Wait for 3 seconds to post again
	// read the status code and body of the response
	statusCode = httpClient.responseStatusCode();
	response = httpClient.responseBody();
	
	Serial.print("Status code: ");
	Serial.println(statusCode);
	Serial.print("Response: ");
	Serial.println(response);
	
}

void postToServer(JsonObject data) {
	dataStr = "";
	serializeJson(data, dataStr);
	Serial.println(dataStr);
	httpClient.beginRequest();
	httpClient.post(path);
	httpClient.sendHeader("Content-Type", "application/json");
	httpClient.sendHeader("Content-Length", dataStr.length());
	httpClient.beginBody();
	httpClient.print(dataStr);
	httpClient.endRequest();
}
\end{lstlisting}
