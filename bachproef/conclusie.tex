%%=============================================================================
%% Conclusie
%%=============================================================================

\chapter{Conclusie}
\label{ch:conclusie}

% TODO: Trek een duidelijke conclusie, in de vorm van een antwoord op de
% onderzoeksvra(a)g(en). Wat was jouw bijdrage aan het onderzoeksdomein en
% hoe biedt dit meerwaarde aan het vakgebied/doelgroep? 
% Reflecteer kritisch over het resultaat. In Engelse teksten wordt deze sectie
% ``Discussion'' genoemd. Had je deze uitkomst verwacht? Zijn er zaken die nog
% niet duidelijk zijn?
% Heeft het onderzoek geleid tot nieuwe vragen die uitnodigen tot verder 
%onderzoek?

Dit onderzoek geeft een antwoord op de onderzoeksvraag 'is het mogelijk om een economisch rendabel GPS-toestel te ontwikkelen dat niet bezwijkt onder ongunstige omstandigheden?'.
\newline
\newline
Voor het onderzoek is er een literatuurstudie uitgevoerd om de juiste locatiebepalingstechnieken te bepalen. Ook zijn de vereisten hier naar boven gekomen waaraan de proof of concept moest voldoen. 
\newline
\newline
De proof of concept slaagt slechts gedeeltelijk, want uit de requirementanalyse blijkt dat de prijs hoger ligt dan wat respondenten er maximum aan zouden willen uitgeven. Een mogelijke verklaring hiervoor is dat de ondervraagden niet volledig weten wat de mogelijkheden van een GPS-tracker op basis van een webapplicatie zijn, aangezien 42,9\% van de respondenten niet bekend is met dit concept. Toch is de prijs veel lager dan de concurrentie en de vooropgestelde maximumprijs van 200 euro. In het onderzoek kon een waterdichte, nauwkeurige GPS-tracker gecreëerd worden voor \textbf{166,14}. Uit de testen bleek dat de proof of concept resistent is tegen ongunstige omstandigheden. 
\newline
\newline
Indien er gekeken wordt naar de deelonderzoeksvragen, los van de requirementanalyse, slaagt de proof of concept.
\begin{itemize}
	\item De proof of concept is in staat zijn locatie te delen.
	\item De proof of concept is goedkoper dan een smartphone (200 euro).
	\item De proof of concept is waterbestendig.
	\item De proof of concept blijft functioneel werken onder water tot een diepte van 0,25 meter.
	\item De proof of concept werkt accurater dan een ingebouwde GPS van een gsm.
	\item De proof of concept kan getracked worden aan de hand van een webapplicatie en is gebruiksvriendelijk (subjectief).
\end{itemize}
De deelonderzoeksvraag 'is de proof of concept zoutbestendig?' kon niet onderzocht worden wegens de coronapandemie. Ook de functionaliteit van de proof of concept onder sneeuw kon niet onderzocht worden. Er kan vermoed worden dat de proof of concept ook blijft functioneren onder een sneeuwlaag van meer dan 0,25 meter omdat de massadichtheid van water hoger ligt dan die van sneeuw. Hoe de GPS-tracker functioneert in zout water, is vooralsnog onbekend.
\newline
De proof of concept slaagt voor alle (onderzochte) onderzoeksvragen. Het is derhalve bewezen dat het mogelijk is om een gebruiksvriendelijk, economisch rendabel GPS-toestel te ontwikkelen dat niet bezwijkt onder ongunstige omstandigheden.