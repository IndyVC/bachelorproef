%%=============================================================================
%% Inleiding
%%=============================================================================

\chapter{\IfLanguageName{dutch}{Inleiding}{Introduction}}
\label{ch:inleiding}
Het Global Positioning System (GPS) is een gevolg van de technologische revolutie. Het heeft één van de meest verspreide ICT-toepassingen als systeem. Dit systeem kan gebruikt worden door verschillende toestellen en voor verschillende doeleinden. Bijvoorbeeld; een route uitstippelen met behulp van je gsm, GPS-ontvangers die je in staat stellen om zaken terug te vinden en avalanche trackers (lawinepieper). Geschat wordt dat er 3.5 miljard mensen een gsm bezitten, wat het gebruik van het Global Positioning System zeer toegankelijk maakt. \autocite{numberOfSmartphones}. Ook worden lawinepiepers vaak gebruikt tijdens het wintersporten. Dit toestel wordt gebruikt om mensen op te sporen die bedolven worden onder een lawine. Uniek aan dit toestel is dat ze werken op een aparte frequentie (457 kHz). Een lawinepieper kost gemiddeld een €200, wat een behoorlijk bedrag is. \cite{avalancheTransceivers}
Een nieuwe toepassing zou kunnen zijn: het opsporen van materiaal onder water, een hele uitdaging gezien een GPS-signaal moeilijk opgevangen wordt onder water. \autocite{underwaterProblem} De oorzaak hiervan is dat satellieten moeilijk de elektromagnetische signalen van onder het wateroppervlakte kunnen opvangen. Ook kosten dergelijke toestellen met een goede gps relatief veel geld, wat het niet toegankelijk maakt voor een winter- of watersporter. 
Een andere factor waarmee rekening gehouden moet worden bij het aankopen van een gps zijn de ongunstige omstandigheden. Zo moet de casing van een gps water- en weerbestendig zijn.
In het kader van dit onderzoek werd er een proof of concept (PoC) gerealiseerd die getest werd in ongunstige omstandigheden. Hierbij werd er gekeken of het waterdicht en goedkoop is. Er werd ook een applicatie ontwikkeld om de tracker op te volgen.


\section{\IfLanguageName{dutch}{Probleemstelling}{Problem Statement}}
\label{sec:probleemstelling}

 Een GPS-toestel is duur, moeilijk te configureren en niet gebruiksvriendelijk. Bij watersport stelt zich een bijkomend probleem, er bestaat momenteel geen budgetvriendelijk toestel dat in staat is om verloren surfboards te lokaliseren. Dit onderzoek focust zich op winter- en watersporters die nood hebben aan een betrouwbaar, goedkoop, waterdicht en gebruiksvriendelijke GPS-toestel.

\section{\IfLanguageName{dutch}{Onderzoeksvraag}{Research question}}
\subsection{\IfLanguageName{dutch}{Hoofdonderzoeksvraag}{Main research question}}
\label{sec:onderzoeksvraag}

Zoals reeds aangegeven zal dit onderzoek zich vooral focussen op het ontwerpen van een goedkoop GPS-toestel dat goed presteert in ongunstige omstandigheden. Hieruit volgt dan de volgende hoofdonderzoeksvraag:
\newline
\begin{itemize}
	\item[] \textit{Is het mogelijk om een goedkoop GPS-toestel te ontwikkelen dat niet bezwijkt onder ongunstige omstandigheden?}
\end{itemize}

\subsection{\IfLanguageName{dutch}{Deelonderzoeksvragen}{Sub-research question}}
Als resultaat van dit onderzoek moet er een proof of concept (PoC) ontworpen worden dat voldoet aan de volgende vereisten:
\begin{itemize}
	\item Proof of concept moet in staat zijn om zijn locatie te delen;
	\item Proof of concept moet goedkoop zijn;
	\item Proof of concept moet zoutbestendig zijn;
	\item Proof of concept moet waterdicht zijn;
    \item Proof of concept moet functioneel blijven werken  onder water en sneeuw;
	\item Proof of concept moet minstens even accuraat werken als een ingebouwde GPS in een gsm;
	\item Proof of concept moet getracked kunnen worden met behulp van een applicatie;
	\item Applicatie moet gebruiksvriendelijk zijn.
\end{itemize}

\section{\IfLanguageName{dutch}{Onderzoeksdoelstelling}{Research objective}}
\label{sec:onderzoeksdoelstelling}
Het hoofddoel is een proof of concept ontwerpen dat toegankelijk is voor water- en wintersporters. Deze doelgroep heeft nood aan een GPS-systeem dat goed blijft functioneren in alle omstandigheden. Het onderzoek is geslaagd wanneer aan alle vereisten voldaan worden.


\section{\IfLanguageName{dutch}{Opzet van deze bachelorproef}{Structure of this bachelor thesis}}
\label{sec:opzet-bachelorproef}

% Het is gebruikelijk aan het einde van de inleiding een overzicht te
% geven van de opbouw van de rest van de tekst. Deze sectie bevat al een aanzet
% die je kan aanvullen/aanpassen in functie van je eigen tekst.

Het vervolg van deze bachelorproef is opgebouwd als volgt:

In Hoofdstuk~\ref{ch:stand-van-zaken} wordt een overzicht gegeven van de stand van zaken binnen het onderzoeksdomein, op basis van een literatuurstudie.

In Hoofdstuk~\ref{ch:methodologie} wordt de methodologie toegelicht en worden de gebruikte onderzoekstechnieken besproken om een antwoord te kunnen formuleren op de onderzoeksvragen.

% TODO: Vul hier aan voor je eigen hoofstukken, één of twee zinnen per hoofdstuk

In Hoofdstuk~\ref{ch:corpus} vindt u het volledig bekomen resultaat (proof of concept met bijhorende applicatie).
\newline
\newline
In Hoofdstuk~\ref{ch:conclusie}, tenslotte, wordt de conclusie gegeven en een antwoord geformuleerd op de onderzoeksvragen.