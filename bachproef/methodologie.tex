%%=============================================================================
%% Methodologie
%%=============================================================================

\chapter{\IfLanguageName{dutch}{Methodologie}{Methodology}}
\label{ch:methodologie}

%% TODO: Hoe ben je te werk gegaan? Verdeel je onderzoek in grote fasen, en
%% licht in elke fase toe welke stappen je gevolgd hebt. Verantwoord waarom je
%% op deze manier te werk gegaan bent. Je moet kunnen aantonen dat je de best
%% mogelijke manier toegepast hebt om een antwoord te vinden op de
%% onderzoeksvraag.
Dit onderzoek is vooraf gegaan met een literatuurstudie over GPS en alle andere mogelijke lokatiebepaling technieken. De literatuurstudie is terug te vinden in hoofdstuk \ref{ch:stand-van-zaken}.
\newline
\newline
Na de literatuurstudie volgt het ontwikkelen van de proof of concept, webapplicatie en de backend. In hoofdstuk \ref{ch:proof-of-concept} wordt er besproken hoe de proof of concept in elkaar zit. Voor het script is er gebruik gemaakt van de code editor van Arduino zelf. 
\newline
\newline
De data van de proof of concept wordt verwerkt in een backend die online geplaatst is op \href{www.heroku.com}{Heroku}. Deze backend wordt besproken in hoofdstuk \ref{ch:backend}.
\newline
\newline
Als volgt wordt de data die ontvangen is weergegeven in een webapplicatie (hoofdstuk \ref{ch:frontend}). Dit zorgt ervoor dat de proof of concept makkelijk te gebruiken is. 
\newline
\newline
Na het ontwikkelen van de hele applicatie en proof of concept werden verschillende experimenten en testen uitgevoerd. Deze resultaten zijn terug te vinden in hoofdstuk \ref{ch:resultaten}.


