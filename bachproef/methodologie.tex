%%=============================================================================
%% Methodologie
%%=============================================================================

\chapter{\IfLanguageName{dutch}{Methodologie}{Methodology}}
\label{ch:methodologie}

%% TODO: Hoe ben je te werk gegaan? Verdeel je onderzoek in grote fasen, en
%% licht in elke fase toe welke stappen je gevolgd hebt. Verantwoord waarom je
%% op deze manier te werk gegaan bent. Je moet kunnen aantonen dat je de best
%% mogelijke manier toegepast hebt om een antwoord te vinden op de
%% onderzoeksvraag.
Het onderzoek is begonnen met een literatuurstudie over het Global Positioning System en alle andere lokatiebepalingstechnieken. De literatuurstudie is terug te vinden in hoofdstuk \ref{ch:stand-van-zaken}.
\newline
\newline
Na de literatuurstudie volgt het ontwikkelen van de proof of concept, de webapplicatie en de backend. In sectie \ref{ch:proof-of-concept} wordt er besproken hoe de proof of concept ontworpen is. Voor het script is er gebruik gemaakt van de code editor van Arduino zelf. 
\newline
\newline
De data van de proof of concept wordt verwerkt in een backend die online geplaatst is op \href{www.heroku.com}{Heroku}. Deze backend wordt besproken in sectie \ref{ch:backend}.
\newline
\newline
Vervolgens wordt de data die ontvangen is weergegeven in een webapplicatie (sectie \ref{ch:frontend}). Dit zorgt ervoor dat de proof of concept gebruiksvriendelijk is. 
\newline
\newline
Na het ontwikkelen van de hele applicatie en de proof of concept werden verschillende experimenten en testen uitgevoerd. Deze resultaten zijn terug te vinden in hoofdstuk \ref{ch:resultaten}.
\section{\IfLanguageName{dutch}{Requirementanalyse}{Requirement analysis}}
\label{ch:requirementAnalyse}
Voor dit onderzoek is er een requirementanalyse uitgevoerd om te weten te komen of de proof of concept een meerwaarde kan bieden aan ervaren snowboarders en/of surfers. De requirementanalyse is uitgevoerd aan de hand van een enquête. Deze enquête bestaat uit de volgende vragen:
\begin{itemize}
	\item Weet u wat een GPS-tracker is? Hebt u reeds overwogen om een GPS-tracker aan te schaffen?
	\item Heeft u ooit al gehoord van een gelijkaardige GPS-tracker die gebruik maakt van een webapplicatie?
	\item Heeft u zelf al een GPS-tracker gebruikt die u kan tracken aan de hand van een webapplicatie?
	\item Indien u reeds een GPS-tracker heeft gebruikt, hoeveel heeft het gekost? Zoniet, Hoeveel zou u maximum uitgeven aan een GPS-tracker?
	\item Aan welke voorwaarden zou de GPS-tracker moeten voldoen? Score van 0 (minst belangrijk) tot en met 5 (meest belangrijk).
	\begin{itemize}
		\item Accuraatheid
		\item Prijs
		\item Waterbestendigheid
		\item Levensduur batterij
		\item Eenvoud van het tracken
	\end{itemize}
	\item Hoe lang moet de batterij minstens meegaan?
	\item Zou u de webapplicatie alleen gebruiken in nood of ook om activiteiten te delen op sociale media? (Runkeeper, ...)
	\item Zou u zichzelf veiliger voelen als u gebruik maakt van een GPS-tracker tijdens het snowboarden/surfen?
\end{itemize}
De resultaten van de requirementanalyse worden besproken in Hoofdstuk~\ref{ch:resultaten}.


