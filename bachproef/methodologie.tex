%%=============================================================================
%% Methodologie
%%=============================================================================

\chapter{\IfLanguageName{dutch}{Methodologie}{Methodology}}
\label{ch:methodologie}

%% TODO: Hoe ben je te werk gegaan? Verdeel je onderzoek in grote fasen, en
%% licht in elke fase toe welke stappen je gevolgd hebt. Verantwoord waarom je
%% op deze manier te werk gegaan bent. Je moet kunnen aantonen dat je de best
%% mogelijke manier toegepast hebt om een antwoord te vinden op de
%% onderzoeksvraag.
Het onderzoek is begonnen met een literatuurstudie over het Global Positioning System en alle andere locatiebepalingstechnieken. De literatuurstudie is terug te vinden in hoofdstuk \ref{ch:stand-van-zaken}.
\newline
\newline
Na de literatuurstudie volgt het ontwikkelen van de proof of concept, de webapplicatie, de backend en de mobiele applicatie. In sectie~\ref{ch:proof-of-concept} wordt er besproken hoe de proof of concept ontworpen is. Voor het script is er gebruik gemaakt van de code-editor van Arduino. 
\newline
\newline
De data van de proof of concept wordt verwerkt in een backend die online geplaatst is op \href{www.heroku.com}{Heroku}. Deze backend wordt besproken in sectie~\ref{ch:backend}.
\newline
\newline
Vervolgens wordt de data die ontvangen is weergegeven in een webapplicatie (sectie~\ref{ch:frontend}). Dit zorgt ervoor dat de proof of concept gebruiksvriendelijk is. 
\newline
\newline
Na het ontwikkelen van de hele applicatie en de proof of concept werden verschillende experimenten en testen uitgevoerd. Deze resultaten zijn terug te vinden in hoofdstuk \ref{ch:resultaten}.
\section{\IfLanguageName{dutch}{Requirementanalyse}{Requirement analysis}}
\label{ch:requirementAnalyse}
Voor dit onderzoek is er een requirementanalyse uitgevoerd om te weten te komen of de proof of concept een meerwaarde kan bieden voor ervaren sneeuw- en/of watersporters. De requirementanalyse is uitgevoerd aan de hand van een enquête (zie bijlage~\ref{ch:enquete}).
\newline
De enquête onderzocht de vereisten en verwachtingen van de proof of concept.
De resultaten van de requirementanalyse worden besproken in Hoofdstuk~\ref{ch:resultaten}.


