%%=============================================================================
%% Inleiding
%%=============================================================================

\chapter{\IfLanguageName{dutch}{Inleiding}{Introduction}}
\label{ch:inleiding}
Veel onderzoek stelt dat de huidige tijd gekenmerkt wordt door een verstrekkende technologische revolutie. Deze revolutie heeft veel systemische veranderingen teweeg gebracht in de manier waarop mensen hun leefwereld structureren en betekenis geven. Eén van die nieuwe toepassingen is het Global Positition System (GPS) dat mensen in staat stelt om zich waar dan ook op deze planeet te navigeren. Het wordt beschouwd als één van de meest verspreide ICT-toepassingen ter wereld. De opgang van de GSP verloopt quasi simultaan met de verspreiding van de smartphone. Geschat wordt dat er inmiddels 3,5 miljard mensen zijn met een smartphone, hetgeen het alledaagse gebruik van de GPS in verregaande mate gefaciliteerd heeft \autocite{numberOfSmartphones}

Een specifieke toepassing van het GPS-systeem dat in deze paper aan bod komt, is de zogenaamde lawinepieper of avalanche tracker. Deze avalanche trackers worden gebruikt om mensen terug te vinden die bedolven worden door een lawine bij het wintersporten. Uniek aan deze systemen is dat ze opereren op een eigen frequentie (457kHz).

In deze paper zal onderzocht worden of er - naar analogie met de lawinepieper - een soortgelijke toepassing mogelijk is voor het opsporen van vermiste personen onder water. Dit is een uitdaging via de huidige beschikbare technologieën omdat ze weinig efficiënt zijn bij het opsporen van materiaal dat zich onder water bevindt \autocite{underwaterProblem}. Satellieten ondervinden problemen bij het opvangen van elektromagnetische signalen onder het wateroppervlakte. Toestellen die hier wel in slagen, kosten ook veel geld. Een lawinepieper kost bijvoorbeeld gemiddeld tweehonderd euro. Dit zorgt ervoor dat winter- en watersporters vaker bereid zijn om veiligheid in te ruilen voor een goedkopere uitoefening van hun hobby. Tot slot moet de casing van dergelijke gps-toestellen ook water- en weerbestendig zijn. Winter- en watersporten worden vaak uitgeoefend in uitdagende omstandigheden.

In het kader van dit onderzoek werd een proof of concept (PoC) gerealiseerd die getest werd in ongunsitge omstandigheden. Tot slot werd ook een applicatie ontwikkeld om de tracker op te volgen.


\section{\IfLanguageName{dutch}{Probleemstelling}{Problem Statement}}
\label{sec:probleemstelling}

 Water- en wintersporters opteren vaak niet voor een GPS-tracker doordat deze toestellen duur, niet gebruiksvriendelijk en moeilijk te configureren zijn. GPS-trackers kunnen, als watersporter, gebruikt worden om verloren uitrusting terug te vinden. Een andere optie is dat de GPS-tracker gebruikt wordt als lawinepieper.  Dit onderzoek focust zich op winter- en watersporters die nood hebben aan een betrouwbaar, goedkoop, waterdicht en gebruiksvriendelijke GPS-toestel.

\section{\IfLanguageName{dutch}{Onderzoeksvraag}{Research question}}
\subsection{\IfLanguageName{dutch}{Hoofdonderzoeksvraag}{Main research question}}
\label{sec:onderzoeksvraag}

Zoals reeds aangegeven zal dit onderzoek zich vooral focussen op het ontwerpen van een goedkoop GPS-toestel dat goed presteert in ongunstige omstandigheden. Hieruit volgt dan de volgende hoofdonderzoeksvraag:
\newline
\begin{itemize}
	\item[] \textit{Is het mogelijk om een economisch rendabel GPS-toestel te ontwikkelen dat niet bezwijkt onder ongunstige omstandigheden?}
\end{itemize}

\subsection{\IfLanguageName{dutch}{Deelonderzoeksvragen}{Sub-research question}}
Als resultaat van dit onderzoek moet er een proof of concept (PoC) ontworpen worden dat voldoet aan de volgende vereisten:
\begin{itemize}
	\item Is de proof of concept in staat om zijn locatie te delen?
	\item Is de proof of concept goedkoper dan een smarthphone (200 euro)?
	\item Is de proof of concept zoutbestendig?
	\item  Is de proof of concept waterbestendig?
    \item Kan de proof of concept functioneel blijven werken  onder water en sneeuw?
	\item Kan de proof of concept minstens even accuraat werken als een ingebouwde GPS in een gsm?
	\item Kan de proof of concept getracked worden aan de hand van een webapplicatie?
	\item Is de webapplicatie gebruiksvriendelijk?
\end{itemize}

\section{\IfLanguageName{dutch}{Onderzoeksdoelstelling}{Research objective}}
\label{sec:onderzoeksdoelstelling}
Het hoofddoel is een proof of concept ontwerpen dat toegankelijk is voor water- en wintersporters. Deze doelgroep heeft nood aan een GPS-systeem dat goed blijft functioneren in alle omstandigheden. Het onderzoek is geslaagd wanneer aan alle vereisten voldaan worden.


\section{\IfLanguageName{dutch}{Opzet van deze bachelorproef}{Structure of this bachelor thesis}}
\label{sec:opzet-bachelorproef}

% Het is gebruikelijk aan het einde van de inleiding een overzicht te
% geven van de opbouw van de rest van de tekst. Deze sectie bevat al een aanzet
% die je kan aanvullen/aanpassen in functie van je eigen tekst.

Het vervolg van deze bachelorproef is opgebouwd als volgt:

In Hoofdstuk~\ref{ch:stand-van-zaken} wordt een overzicht gegeven van de stand van zaken binnen het onderzoeksdomein, op basis van een literatuurstudie.

In Hoofdstuk~\ref{ch:methodologie} wordt de methodologie toegelicht en worden de gebruikte onderzoekstechnieken besproken om een antwoord te kunnen formuleren op de onderzoeksvragen.

% TODO: Vul hier aan voor je eigen hoofstukken, één of twee zinnen per hoofdstuk

In Hoofdstuk~\ref{ch:corpus} en Hoofdstuk~\ref{ch:resultaten} vindt u het volledig bekomen resultaat (proof of concept met bijhorende applicatie).
\newline
\newline
In Hoofdstuk~\ref{ch:conclusie}, tenslotte, wordt de conclusie gegeven en een antwoord geformuleerd op de onderzoeksvragen.