%%=============================================================================
%% Inleiding
%%=============================================================================

\chapter{\IfLanguageName{dutch}{Inleiding}{Introduction}}
\label{ch:inleiding}
Een van de meest verspreide ICT-toepassingen, als gevolg van de technologische revolutie, is het Global Positioning System (GPS). Dit systeem kan gebruikt worden door verschillende toestellen voor verschillende doeleinden. Een voorbeeld hiervan is een route uitstippelen met behulp van je gsm. Naar schatting zouden er 3.5 miljard mensen een gsm bezitten, wat het gebruik van het Global Positioning System zeer toegankelijk maakt. \autocite{numberOfSmartphones} Er bestaan verschillende toestellen met elk hun eigen toepassing. Een voorbeeld hiervan is de lawinepieper. Dit toestel wordt gebruikt om mensen op te sporen die bedolven zijn onder een lawine. Wat uniek is aan dit toestel is dat ze werken op een aparte frequentie (457 kHz), maar voor een lawinepieper betaal je al rap ongeveer 200 euro. \cite{avalancheTransceivers}
Een andere toepassing zou kunnen zijn: het opsporen van verloren goederen onder of in water. Helaas is dit een hele uitdaging doordat
het GPS-signaal niet optimaal werkt onder water. \cite{underwaterProblem} Dit komt doordat de elektromagnetische signalen van de satellieten het water niet of amper kunnen penetreren. Ook kosten dergelijke toestellen met een goede gps relatief veel geld, wat het niet toegankelijk maakt voor een winter- of watersporter. 
Een andere factor waarmee rekening gehouden moet worden bij het aankopen van een gps zijn ongunstige omstandigheden. Zo moet de casing van een gps resistent zijn tegen (zout)water en sneeuw.
In het kader van dit onderzoek werd er een proof of concept (PoC) gecreëerd die getest werd in ongunstige omstandigheden. Hierbij werd er gekeken of het waterdicht en goedkoop is. Ook werd er ook een applicatie ontwikkeld om de tracker op te volgen.


\section{\IfLanguageName{dutch}{Probleemstelling}{Problem Statement}}
\label{sec:probleemstelling}

GPS-toestellen kunnen duur oplopen. Als je kijkt naar een lawinepieper betaal je al rap boven de 200 euro. Ook zijn deze toestellen moeilijk te configureren of te gebruiken.Een ander probleem vindt plaats bij de watersport. Er bestaat momenteel geen toestel dat in staat is om verloren surfboards te lokaliseren. Dit onderzoek focust zich op winter- en watersporters die nood hebben aan een betrouwbaar, goedkoop, waterdicht en makkelijk te gebruiken GPS-toestel.

\section{\IfLanguageName{dutch}{Onderzoeksvraag}{Research question}}
\subsection{\IfLanguageName{dutch}{Hoofdonderzoeksvraag}{Main research question}}
\label{sec:onderzoeksvraag}

Zoals reeds aangegeven zal dit onderzoek zich vooral focussen op het opleveren van een goedkoop GPS-toestel dat goed presteert in ongunstige omstandigheden. Hieruit volgt dan de volgende hoofdonderzoeksvraag.
\newline
\begin{itemize}
	\item Is het mogelijk om een goedkoop GPS-toestel te ontwikkelen die niet bezwijkt onder ongunstige omstandigheden?
\end{itemize}

\subsection{\IfLanguageName{dutch}{Deelonderzoeksvragen}{Sub-research question}}
Als resultaat van dit onderzoek moet er een proof of concept (PoC) opgeleverd worden die voldoet aan de volgende vereisten:
\begin{itemize}
	\item Proof of concept moet in staat zijn om zijn locatie te delen.
	\item Proof of concept moet goedkoop zijn.
	\item Proof of concept moet resistant zijn tegen zout.
	\item Proof of concept moet waterdicht zijn.
    \item Proof of concept moet functioneel blijven werken  onder water en sneeuw.
	\item Proof of concept moet minstens even accuraat werken als een ingebouwde GPS in een gsm.
	\item Proof of concept moet getracked kunnen worden met behulp van een applicatie
	\item Applicatie moet gebruiksvriendelijk zijn
\end{itemize}

\section{\IfLanguageName{dutch}{Onderzoeksdoelstelling}{Research objective}}
\label{sec:onderzoeksdoelstelling}
Het hoofddoel is een proof of concept opleveren die toegankelijk is voor water- en wintersporters die nood hebben aan een GPS-systeem dat goed blijft functioneren in ongunstige omstandigheden. Dit onderzoek is geslaagd wanneer het hoofddoel behaald wordt.


\section{\IfLanguageName{dutch}{Opzet van deze bachelorproef}{Structure of this bachelor thesis}}
\label{sec:opzet-bachelorproef}

% Het is gebruikelijk aan het einde van de inleiding een overzicht te
% geven van de opbouw van de rest van de tekst. Deze sectie bevat al een aanzet
% die je kan aanvullen/aanpassen in functie van je eigen tekst.

De rest van deze bachelorproef is als volgt opgebouwd:

In Hoofdstuk~\ref{ch:stand-van-zaken} wordt een overzicht gegeven van de stand van zaken binnen het onderzoeksdomein, op basis van een literatuurstudie.

In Hoofdstuk~\ref{ch:methodologie} wordt de methodologie toegelicht en worden de gebruikte onderzoekstechnieken besproken om een antwoord te kunnen formuleren op de onderzoeksvragen.

% TODO: Vul hier aan voor je eigen hoofstukken, één of twee zinnen per hoofdstuk

In Hoofdstuk~\ref{ch:corpus} vindt u het volledig bekomen resultaat (proof of concept met bijhorende applicatie)
.
In Hoofdstuk~\ref{ch:conclusie}, tenslotte, wordt de conclusie gegeven en een antwoord geformuleerd op de onderzoeksvragen.