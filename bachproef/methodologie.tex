%%=============================================================================
%% Methodologie
%%=============================================================================

\chapter{\IfLanguageName{dutch}{Methodologie}{Methodology}}
\label{ch:methodologie}

%% TODO: Hoe ben je te werk gegaan? Verdeel je onderzoek in grote fasen, en
%% licht in elke fase toe welke stappen je gevolgd hebt. Verantwoord waarom je
%% op deze manier te werk gegaan bent. Je moet kunnen aantonen dat je de best
%% mogelijke manier toegepast hebt om een antwoord te vinden op de
%% onderzoeksvraag.
Het onderzoek is begonnen met een literatuurstudie over het Global Positioning System en alle andere lokatiebepalingstechnieken. De literatuurstudie is terug te vinden in hoofdstuk \ref{ch:stand-van-zaken}.
\newline
\newline
Na de literatuurstudie volgt het ontwikkelen van de proof of concept, de webapplicatie en de backend. In sectie~\ref{ch:proof-of-concept} wordt er besproken hoe de proof of concept ontworpen is. Voor het script is er gebruik gemaakt van de code editor van Arduino zelf. 
\newline
\newline
De data van de proof of concept wordt verwerkt in een backend die online geplaatst is op \href{www.heroku.com}{Heroku}. Deze backend wordt besproken in sectie~\ref{ch:backend}.
\newline
\newline
Vervolgens wordt de data die ontvangen is weergegeven in een webapplicatie (sectie~\ref{ch:frontend}). Dit zorgt ervoor dat de proof of concept gebruiksvriendelijk is. 
\newline
\newline
Na het ontwikkelen van de hele applicatie en de proof of concept werden verschillende experimenten en testen uitgevoerd. Deze resultaten zijn terug te vinden in hoofdstuk \ref{ch:resultaten}.
\section{\IfLanguageName{dutch}{Requirementanalyse}{Requirement analysis}}
\label{ch:requirementAnalyse}
Voor dit onderzoek is er een requirementanalyse uitgevoerd om te weten te komen of de proof of concept een meerwaarde kan bieden aan ervaren sneeuw- en/of watersporters. De requirementanalyse is uitgevoerd aan de hand van een enquête. Deze enquête bestaat uit de volgende vragen:
\begin{enumerate}
	\item Weet u wat een GPS-tracker is?
		\begin{itemize}
			\item Ja, ik weet wat een GPS-tracker is
			\item Nee, ik ken het begrip niet
			\item Ja, ik heb er reeds van gehoord maar kan het begrip nog niet 100 percent duiden
		\end{itemize}
	\item Heeft u reeds overwogen om een GPS-tracker aan te schaffen?
		\begin{itemize}
			\item Ja
			\item Nee
		\end{itemize}
	\item Heeft u ooit al gehoord van een GPS-tracker die gebruik maakt van een applicatie? Zoals 'Runkeeper' maar in je webbrowser
		\begin{itemize}
			\item Ja, ik heb er ervaring mee
			\item Ja, ik heb er al van gehoord maar nog nooit één gebruikt
			\item Nee, ik heb er geen ervaring mee
		\end{itemize}
	\item Indien u reeds een GPS-tracker heeft aangekocht, hoeveel heeft deze gekost?
		\begin{itemize}
			\item 0 - 50 euro
			\item 51 - 100 euro
			\item 101 - 150 euro
			\item 151 - 200 euro
			\item Meer dan 200 euro
		\end{itemize}
	\item Hoeveel zou u maximum besteden aan een GPS-tracker?
	\item Hoe belangrijk vindt u de accuraatheid van de locatiebepaling? Van weinig belangrijk (1) tot heel belangrijk (10).
	\item Hoe belangrijk vindt u, als watersporter/sneeuwsporter, de waterbestendigheid van een GPS-tracker die u zou gebruiken tijdens het sporten? Van weinig belangrijk (1) tot heel belangrijk (10).
	\item Hoe belangrijk vindt u de gebruiksvriendelijkheid van het tracken?Van weinig belangrijk (1) tot heel belangrijk (10).
	\item Hoelang vindt u dat de GPS-tracker minstens moet blijven werken? Uitdrukken in aantal uren
		\begin{itemize}
			\item 0-6
			\item 6-12
			\item 12-18
			\item 18-24
		\end{itemize}
	\item Zou u de webapplicatie, die u toegang biedt tot het tracken, alleen gebruiken in nood en/of om activiteiten te delen op sociale media? (Bijvoorbeeld Runkeeper, ... )
		\begin{itemize}
			\item Alleen in noodgevallen
			\item Alleen om activiteiten te delen op sociale media, ...
			\item Beide
		\end{itemize}
	\item  Zou u een budgetvriendelijke GPS-tracker (max. 150 euro) aankopen indien deze op de markt komt?
		\begin{itemize}
			\item Ja
			\item Misschien
			\item Nee
		\end{itemize}
	\item Indien u een budgetvriendelijke GPS-tracker aankoopt, voor welke sporten zou u deze gebruiken?
		\begin{itemize}
			\item Watersport
			\item Sneeuwsport
			\item Allebei
		\end{itemize}

\end{enumerate}
De resultaten van de requirementanalyse worden besproken in Hoofdstuk~\ref{ch:resultaten}.


