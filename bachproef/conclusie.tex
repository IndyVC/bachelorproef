%%=============================================================================
%% Conclusie
%%=============================================================================

\chapter{Conclusie}
\label{ch:conclusie}

% TODO: Trek een duidelijke conclusie, in de vorm van een antwoord op de
% onderzoeksvra(a)g(en). Wat was jouw bijdrage aan het onderzoeksdomein en
% hoe biedt dit meerwaarde aan het vakgebied/doelgroep? 
% Reflecteer kritisch over het resultaat. In Engelse teksten wordt deze sectie
% ``Discussion'' genoemd. Had je deze uitkomst verwacht? Zijn er zaken die nog
% niet duidelijk zijn?
% Heeft het onderzoek geleid tot nieuwe vragen die uitnodigen tot verder 
%onderzoek?

Dit onderzoek geeft een antwoord op de onderzoeksvraag 'Is het mogelijk om een economisch rendabel GPS-toestel te ontwikkelen dat niet bezwijkt onder ongunstige omstandigheden?'
\newline
\newline
Voor het onderzoek is er een literatuurstudie uitgevoerd om de juiste lokatiebepalingstechnieken te bepalen. Ook zijn de vereisten hier naar boven gekomen waaraan de proof of concept moest voldoen. 
\newline
\newline
De proof of concept slaagt slechts gedeeltelijk, want uit de requirementanalyse blijkt dat de prijs hoger ligt dan wat correspondenten er maximum aan willen uitgeven. Een mogelijke verklaring hiervoor is dat de ondervraagden niet volledig weten aan wat ze zich konden verwachten, want 42,9\% wist niet wat de webapplicatie inhield. Toch is de prijs veel lager dan de concurrentie. Voor \textbf{166,14} euro is er een waterdichte, nauwkeurige GPS-tracker opgeleverd die eveneens ook opgevolgd kan worden aan de hand van een, makkelijk te gebruiken, webapplicatie. Uit de testen bleek dat de proof of concept resistent is tegen ongunstige omstandigheden. 
\newline
\newline
Indien er gekeken wordt naar de deelonderzoeksvragen, los van de requirementanalyse, slaagt de proof of concept. De proof of concept is:
\begin{itemize}
	\item De proof of concept is in staat zijn locatie te delen.
	\item De proof of concept is goedkoper dan een smarthphone (200 euro).
	\item De proof of concept is waterbestendig.
	\item De proof of concept blijft functioneel werken onder water tot een diepte van 0,25 meter.
	\item De proof of concept werkt accurater dan een ingebouwde GPS van een gsm.
	\item De proof of concept kan getracked worden aan de hand van een webapplicatie en is gebruiksvriendelijk (subjectief).
\end{itemize}
De deelonderzoeksvraag 'Is de proof of concept zoutbestendig?' kon niet onderzocht worden wegens de corona pandemie. Eveneens kon de functionaliteit van de proof of concept onder sneeuw ook niet onderzocht worden. Deze zaken zijn nog niet duidelijk, maar zouden vermoedelijk ook slagen, aangezien de proof of concept blijft functioneren tot een diepte van 0,25 meter. Dit wijst aan dat de proof of concept \textbf{minstens} tot 0,25 meter kan functioneren onder sneeuw. 
\newline
De proof of concept slaagt voor alle (onderzochte) onderzoeksvragen. Het is bewezen dat het mogelijk is om een gebruiksvriendelijk, economisch rendabel GPS-toestel te ontwikkelen dat niet bezwijkt onder ongunstige omstandigheden.